% dvips -t letter syllabus.dvi -o syllabus.ps ; ps2pdf syllabus.ps
\documentclass[10pt,titlepage,fleqn]{article}

\usepackage{amsmath}
\usepackage{amssymb}
\usepackage{latexsym}
%\usepackage[round]{natbib}
\usepackage{xspace}
\usepackage{graphicx}
%\usepackage{epsfig}

%\usepackage{fancyhdr}
%\pagestyle{fancy}

\usepackage{color}

%=====================================================
%       SPACING COMMANDS (Latex Companion, p. 52)
%=====================================================

\usepackage{setspace}

\renewcommand{\baselinestretch}{1.1}

\textwidth 460pt
\textheight 700pt
\oddsidemargin 0pt
\evensidemargin 0pt

% see Latex Companion, p. 85
\voffset     -50pt
\topmargin     0pt
\headsep      20pt
\headheight    0pt
\footskip     30pt
\hoffset       0pt

\include{carlcommands}

%\newcommand{\figlocA}[1]{/home/carltape/gmt/socal_2005/#1}

\graphicspath{
  {figures/}
}

%=====================================================
\begin{document}
%=====================================================

\begin{tabular}{cc}
\includegraphics[width=8cm]{/home/admin/share/datalib/logos/UAF/UAF/UAFLogo_A_black_horiz.eps} &
\end{tabular}

\bigskip\noindent
{\bf \em QUICK REFERENCE: Section 8 contains the calendar of topics and deadlines.}

\medskip\noindent
Last compiled: \today

\begin{enumerate}
%------------
\item {\bf Course information.}

% NEED TO COORDINATE THIS CLASS TIME WITH:
% FRANZ MICROWAVE REMOTE SENSTING GEOS 657
% MARTIN ICE PHYSICS GEOS 614

\begin{tabular}{ll}
GEOS F627  & {\bf Inverse Problems and Parameter Estimation}, 3 credits (2+3), Spring 2017 \\
%Lab time: & Tuesday, 9:45--12:45 \\
%Lecture time: & Thursday, 9:45--11:45 \\
Lecture: & Monday, 12:00--14:00, 301 Elvey \\
Lab: & Wednesday, 14:00--17:00, 310 Elvey \\
%Meeting location: & TBD \\ %301N Elvey (Geophysical Institute) \\
Prerequisites: & MATH 202 (Calculus III) and MATH 314 (Linear Algebra) %; or permission of instructor
\end{tabular}

%------------
\item {\bf Instructor information.}

\begin{tabular}{ll}
Instructor: & {\bf Carl Tape} \\
Office: & 413D Elvey (Geophysical Institute) \\
Email: & \verb+ctape@alaska.edu+ \\
Phone: & (907) 474-5456 \\
Office hours: & by appointment \\
\end{tabular}

%------------
\item {\bf Course materials.}


\begin{enumerate}

\item {\bf Textbooks.} The required (R) and supplemental (S) textbooks are listed in the following table; bibliographic details are listed at the end of this syllabus. ``Software'' lists the software (if any) used in examples within each book. The Geophysical Institute's Mather library is located on the gound floor of the IARC/Akasofu building.

\begin{tabular}{l|c|c|c|c|c|c|c}
\hline
     &          & &      & \multicolumn{4}{c}{Availability}    \\ \cline{5-8}
Textbook & R & S & Software & UAF       & Mather  &     & UAF     \\
     &          & &          & bookstore & reserve & PDF & e-book  \\ \hline
\cite{AsterE2} Aster           & X & & Matlab & & X & & X \\ \hline
\cite{Tarantola2005} Tarantola & X & & none   & & X & X & \\ \hline
\cite{MenkeE3} Menke           & & X & Matlab & & X & & X \\ \hline
\cite{Parker} Parker           & & X & none   & & X & & \\ \hline
\cite{WeisbergE4} Weisberg     & & X & R      & &   & & X \\ \hline
\end{tabular} \\

\cite{Tarantola2005} can be downloaded as a pdf from Tarantola's webpage. However, he writes: ``Here are the rules: i)~you are invited to download, view, and print the books; ii)~if you work in a commercial company, or in a rich institution (like a university in the developed world), and your plan is to use one of the books from time to time, please purchase it (links below).''

\item Journal articles assigned as reading will be available as PDFs via the UAF google drive. All students must access the drive with their \verb+alaska.edu+ email address.

\item Students will need computers for their homework. General-use computers in UAF labs will be made available to students if needed.

\item Matlab will be the primary computational program for the course.
Matlab is available via a UAF-wide license for graduate students and faculty.
Students are welcome to use alternative programs such as python for homework problems; however several labs and examples are in Matlab.

\end{enumerate}

%------------
\item {\bf Course description.}

An inverse problem is a procedure by which observations or measurements are used with quantitative models to gain inferences about some underlying physical quantity or system. Inverse problems occur in all fields of natural sciences --- even something as simple as fitting a line to scattered data is an inverse problem. This course will provide a general framework, as well as general computational algorithms, for approaching inverse problems. The training should benefit all students in natural sciences who are seeking inferences from data.

{\em Catalog description}: An inverse problem uses observations to infer properties of an unknown physical model. One example is how seismometer recordings can be used to infer the location of an earthquake. This course covers inverse theory and methods for solving inverse problems, including numerous examples arising in the natural sciences. Topics include linear regression, method of least squares, discrete ill-posed inverse problems, estimation of uncertainties, iterative optimization, and probabilistic (Bayesian) and sampling approaches. Assignments and computational laboratory exercises require familiarity with linear algebra and computational tools such as Matlab. 

%A forward problem uses a model to make predictions; an inverse problem uses observations to infer properties of an unknown physical model. One example of an inverse problem is how to use seismometer recordings to infer the location of an earthquake. This course covers inverse theory and methods for solving inverse problems, including numerous examples arising in the natural sciences. Topics include linear regression, method of least squares, discrete ill-posed inverse problems, estimation of uncertainties, iterative optimization, and probabilistic (Bayesian) and sampling approaches. Assignments require familiarity with linear algebra and computational tools such as Matlab.

%------------
\item {\bf Course goals.}

We will explore the ubiquitous realm of inverse problems in Earth sciences: how to use observations to make inferences about underlying physical quantities or processes. Our ultimate goal is to be able to recognize the fundamental components of an inverse problem --- measurements, model parameters, misfit function, forward model --- then to pose an approach to solving the problem, then solve the problem with computational algorithms. Concepts of inverse theory and parameter estimation are fundamental to all observational scientists, which includes most students in the natural sciences. During this course students should acquire both a philosphical and scientific appreciation for inverse methods and problems.

%We will explore the study of earthquakes and Earth's interior structure using seismological theories and algorithms. The underlying physical phenomenon we will examine is the seismic wavefield: the time-dependent, space-dependent elastic waves that originate at an earthquake source (for example, a fault slips) and propagate though the heterogeneous Earth structure, then are finally recorded as time series at seismometers on Earth's surface. Students will examine real seismic data and use computational models to estimate properties about earthquake source and Earth structure. Students will acquire practical, advanced seismological training that will prepare them for seismological investigrations in the future, whether in academic, industry, or government jobs.

%------------
\item {\bf Student learning outcomes.}

Upon completion of this course, students should be able to:
%
\begin{enumerate}
\item Articulate the basic features of forward problems and inverse problems.
\item Describe numerous examples of inverse problems and the basic components of each problem.
\item Set up and solve an inverse problem using the least squares approach.
\item Obtain a linear model from a set of data using multiple linear regression.
\item Understand and use data covariances and model covariances within an inverse problem.
\item Describe singular value decomposition and its relevance to inverse methods.
\item Explain and implement a regularization technique.
\item Explain the importance of sampling algorithms for estimating uncertainties of model parameters.
\item Pose and answer statistical questions from a particular set of model samples.
\item Describe probabilistic approaches to inverse problems.
\item Write, improve, and run computational algorithms in Matlab.
\end{enumerate}

%------------
\item {\bf Instructional methods.}

\begin{enumerate}
\item General course information, assignments, and handouts will be posted on the class website or in the class google drive.
\item Lectures (2 hours per week) will be the primary mode of instruction.
\item Computational laboratory sessions (3 hours per week) include dedicated exercises that provide technical training for homework problems.
%Some lectures will be supplemented with computational examples to prepare students for homework problems.
%\item Each student is expected to lead one extended discussion of a case study of an inverse problem.
\end{enumerate}

%------------
\pagebreak
\item {\bf Course calendar (tentative).}

\hspace{-1.5cm}
\begin{tabular}{cll|l|l|ll}
\hline
  & Day & Date & Topic & Reading & \multicolumn{2}{c}{Homework} \\
  &     &      &       & Due$^\dagger$     & Due & Assigned \\ \hline\hline
%  &   &  & Review of vector calculus & AC & HW-2 & HW-3 \\ 
  &   &  & overview of inverse problems & A1 & --- & HW-1 \\
\hline
  &   &  & review of linear algebra & A-A, \verb+notes_matrix.pdf+ & &  \\
  &   &  & LAB: Linux and Matlab & & & \\
\hline
  &   &  & Taylor series and least squares & \verb+notes_taylor.pdf+ & HW-1 & HW-2 \\
  &   &  & LAB: least squares (\verb+lab_linefit.pdf+) & T3, A-C & & \\
\hline
  &   &  & Taylor series and least squares & \verb+notes_taylor.pdf+ & HW-2 & HW-3 \\
  &   &  & probability density & A-B, \verb+notes_tarantola.pdf+ & & \\
  &   &  & LAB: sampling $\sigma_{\rm M}(\bem)$ (\verb+lab_epi.pdf+) & T2, T7.1 & & \\
\hline
  &   &  & covariance & A-B, \verb+notes_tarantola.pdf+ & HW-3 & HW-4 \\
  &   &  & sampling methods & & & \\
  &   &  & LAB: sampling a N-D $\sigma_{\rm M}(\bem)$ & & & \\
\hline
  &   &  & Homework Review &  & HW-4 & HW-5 \\
  &   &  & LAB: Newton method (\verb+lab_newton.pdf+) & T6.22 & & \\
\hline
  &   &  & generalized least squares & T3, \verb+notes_tarantola.pdf+ & & \\
  &   &  & linear regression & A2 & & \\
  &   &  & LAB: iterative methods (\verb+lab_iter.pdf+) & A6, A9 & & \\
\hline
  &   &  & linear regression & A2 & HW-5 & HW-6 \\
  &   &  & LAB: Aster Ch. 2 & & & \\
\hline
  &   &  & singular value decomposition & A3 & & \\
  &   &  & LAB: Aster Ch. 3 & & HW-6 & \\
\hline\hline
  &   &  & \multicolumn{4}{c}{SPRING BREAK (March 11--19)} \\
  &   &  & \multicolumn{4}{c}{SPRING BREAK (March 11--19)} \\
\hline\hline
&   &  & singular value decomposition & A3 & & HW7 \\
&   &  & LAB: Aster Ch. 3 & & & \\
\hline
&  &  & singular value decomposition & A3 & HW7 & final project \\
&  &  & resolution analysis & A3 & & \\
&  &  & LAB: truncated SVD & & & \\
\hline
&   & & InSAR and parameter estimation: & \verb+lab_mogi.pdf+ & project & HW8 \\
&   &  & \hspace{5pt} volcanoes and earthquakes   & & outline & \\
&   &  & LAB: Mogi source from InSAR & & & \\
\hline
&   & & Tikhonov regularization & A4 & HW8 & final project \\
&   &  & LAB: Aster Ch. 4 & & & \\
\hline
&  &  & discretizing problems with basis functions & A5 & & final project  \\
&  &  & LAB: principal component analysis & & & \\
\hline
&   &  & classical inverse problems & & & final project \\
&   &  & LAB: final project & & & \\
\hline
&   &   & final presentation DUE & & &  \\
&   &   & final report DUE & & & \\ \hline
%& Fri & & & & final report & \\ \hline
%----------------
% & & & Bayesian approach to inverse problems  & T1,A11 & & \\
% & & & Resolution analysis & & & \\
% & & & Principal component analysis & handout & & \\
% & & & LAB: GHW inversion with spherical wavelets & & & \\
\hline
\end{tabular} \\
$^\dagger$A = Ref.~\cite{AsterE2}, T = Ref.~\cite{Tarantola2005} \\
For example, ``A-B'' means Appendix B of Aster's book (Ref \cite{AsterE2}); ``T7,1'' means Section 7.1 of Tarantola's book (Ref \cite{Tarantola2005}).

%\vspace{0.5cm}
\pagebreak
\noindent {\bf Some Important Dates:}

\begin{tabular}{lll}
\hline
%Official first day of instruction:                  & Tuesday & January 17 \\
First class:                                        & Wednesday & January 18 \\
Last day to add class:                              & Friday & January 27 \\
Last day to drop class:                             & Friday & January 27 \\
Last day for student- or faculty-initiated withdraw: & Friday & March 31 \\
Last class:                                         & Wednesday & April 26 \\
Final project report:                               & Wednesday & April 26 \\
Final project presentation:                         & Wednesday & April 26 \\
%Official last day of instruction:                   & Monday & May 1 \\
\hline
\end{tabular}

%------------
%\pagebreak
\item {\bf Course policies.}

\begin{enumerate}
\item {\bf Attendance}: All students are expected to attend and participate in all classes and labs.

%\item {\bf Tardiness}: Students are expected to arrive in class prior to the start of each class. If a student does arrive late, they are expected to do so quietly and inform the instructor without disturbing the class.

\item {\bf Participation and preparation}: Students are expected to come to classes and labs with assigned reading and other assignments completed as noted in the syllabus.

\item {\bf Homework assignments}: 

\begin{enumerate}
\item All assignments are due {\bf at the start of class} on the due date. % noted in the Syllabus.
\item Late assignments will be accepted with a $10\%$ penalty per day late, up to five days late; an assignment that is $\ge 5$ days late will receive a zero. (An assignment that is ``one day late'' would be handed in less than 24 hours after the start time of class on the due date.)
%\item The lowest homework assignment will be dropped when computing the course grade.
\item No digital submission of assigments will be accepted. % (\eg email or Blackboard).
\item The contents within each assignment need to be correctly and clearly ordered and physically connected (\ie stapled, binder clip).
\end{enumerate}

{\bf Homework tips}: Please type or write neatly, keep the solutions in the order assigned and staple pages together. Include only relevant computer output in your solutions (a good approach is to cut and paste the relevant output for each problem into an editor such as Word or Latex). Clearly circle or highlight important numbers in the output, and label them with the question number.

I also suggest that you to include the most relevant portions of your code in your answers, especially in cases when you think your code is not working.  Display numerical answers with a reasonable number of significant figures and with {\em units} if the quantity is not dimensionless.

Homework scores are based on clarity of work, logical progression toward the solution, completeness of interpretation and summaries, and whether a correct solution was obtained.
%I encourage you to discuss homework problems with other students, however the work you turn in must be your own.

%\item {\bf Graded Assignments}: Assignments will be graded for students within seven days of their receipt and returned at the end of the next class.

\item {\bf Labs}: Computational labs are designed to provide technical training relevant for topics in the course and within the homework problems. Students are welcome to work in pairs on the lab exercises.

Labs will be graded on a scale from 0 to 2:
%
\begin{itemize}
\item[0] Student does not participate in lab and does not demonstrate completion of work.
\item[1] Student participates in lab but does not demonstrate completion of work.
\item[2] Student demonstrates completion of work.
\end{itemize}
%
``Completion of work'' will be identified by {\bf a subset of the lab questions}, not the entire lab.
{\em It is the student's responsibility to demonstrate completion of lab work.} This should be done during lab time or during office hour.

\item {\bf Reporting grades}: All student grades, transcripts, and tuition information are available on line at \verb+www.uaonline.alaska.edu+. %If you have difficulty accessing this website, contact the registrar at your local campus.

\item {\bf Consulting fellow students}: Students are permitted to discuss with each other general strategies for particular homework problems. However, the write-up that is handed in---including any computer codes---must be individual work.

\item {\bf Plagiarism:} Students must acknowledge any sources of information---including fellow students---that influenced their homework assignments or final project. Any occurrence of plagiarism will result in forfeiture of all points for the  particular homework assignment. If the plagiarism is between two students, then both students will potentially receive the penalty.

%Furthermore, the UAF catalog states: ``The university may initiate disciplinary action and impose disciplinary sanctions against any student or student organization found responsible for committing, attempting to commit or intentionally assisting in the commission of $\ldots$ cheating, plagiarism, or other forms of academic dishonesty\ldots''

\item All UA student academics and regulations are adhered to in this course. You may find these in the UAF catalog (section ``Academics and Regulations'').

\end{enumerate}


%------------
%\pagebreak
\item {\bf Evaluation.}

\begin{enumerate}
%\item For students in the M.S. or Ph.D. program, you must receive a C (note: not C-) or higher for this course for it to count toward your degree requirements.

\item Grading is based on:

\begin{tabular}{|l|l|}
\hline
5\% & Lab participation \\ \hline
80\% & Homework assignments \\ \hline
15\% & Individual final project \\ \hline
\end{tabular}

\bigskip
\item Overall course grades are based on the following:

\begin{tabular}{|l|c|l|}
\hline
A & $x \ge 93$ & excellent performance:  \\
A-- & $90 \le x < 93$ & student demonstrates deep understanding of the subject \\ \hline
B+ & $87 \le x < 90$ & strong performance: \\ 
B  & $83 \le x < 87$ & student demonstrates strong understanding of the subject, \\ 
B-- & $80 \le x < 83$ & but the work lacks the depth and quality needed for an `A' \\ \hline
C+ & $77 \le x < 80$ & mediocre performance: \\
C  & $73 \le x < 77$ & student demonstrates comprehension of some \\
C-- & $70 \le x < 73$ & essential concepts only \\ \hline
D  & $60 \le x < 70$ & poor performance: \\ 
   &                 & student demonstrates poor comprehension of concepts \\ \hline
F  & $x < 60$ & Failure to complete work with 60\% quality \\ \hline
\end{tabular}

\bigskip
\item {\bf Final Project.} The final project will constitute 15\% of the course grade. The project will involve independent research into an inverse problem of the student's choice. It will require some computation and will be presented in the form of a written report, due on the last lecture class of the semester, and a short in-class presentation during the scheduled final exam. The report will be written in manuscript-submission style and format.
%, using the guidelines for {\em Geophysical Research Letters}.
Additional details, including project suggestions, will be provided by the instructor midway through the course.

%\item {\bf Case study discussions.} Each student will lead a discussion of a case study in a particular inverse problem.
%I will provide a list of suggested questions for discussion, as well as a list of potential topics to choose from.
%Each student must choose a topic that is different from his or her final project topic. Leading discussion will count toward half of the ``Attendance and participation'' category of the course grade.

\end{enumerate}

%------------
\item {\bf Support Services.}

The instructor is available by appointment for additional assistance outside session hours. UAF has many student support programs, including the Math Hotline (1-866-UAF-MATH; 1-866-6284) and the Math and Stat Lab in Chapman building (see \verb+www.uaf.edu/dms/mathlab/+ for hours and details).

%------------
\item {\bf Disabilities Services.}

The Office of Disability Services implements the Americans with Disabilities Act (ADA), and it ensures that UAF students have equal access to the campus and course materials. The Geophysics Program will work with the Office of Disability Services (208 Whitaker, 474-5655) to provide reasonable accommodation to students with disabilities.

%=====================================================

\item {\bf References listed in syllabus.}

\renewcommand{\refname}{}

\vspace{-1.4cm}

\begin{spacing}{1.0}
%\bibliographystyle{agu08}
\bibliographystyle{ieeetr}
\bibliography{preamble,REFERENCES,refs_carl,refs_socal,refs_alaska,refs_ice,refs_source}
\end{spacing}

\end{enumerate}

%=====================================================
\end{document}
%=====================================================
