% dvips -t letter project_guidelines.dvi -o project_guidelines.ps ; ps2pdf project_guidelines.ps
% pdflatex project_guidelines
\documentclass[11pt,titlepage,fleqn]{article}

\usepackage{amsmath}
\usepackage{amssymb}
\usepackage{latexsym}
\usepackage[round]{natbib}
\usepackage{xspace}
\usepackage{epsfig}
\usepackage{bm}

%--------------------------------------------------------------
%       SPACING COMMANDS (Latex Companion, p. 52)
%--------------------------------------------------------------

\usepackage{setspace}    % double-space or single-space

\renewcommand{\baselinestretch}{1.0}

\textwidth 460pt
\textheight 690pt
\oddsidemargin 0pt
\evensidemargin 0pt

% see Latex Companion, p. 85
\voffset     -50pt
\topmargin     0pt
\headsep      20pt
\headheight   15pt
\headheight    0pt
\footskip     30pt
\hoffset       0pt

\include{carlcommands}

\graphicspath{
  {./figures/}
}

%--------------------------------------------------------------
\begin{document}
%-------------------------------------------------------------

\begin{center}
{\large \bf Guidelines for Final Project} \\
GEOS 627: Inverse Problems and Parameter Estimation, Carl Tape \\
%Assigned: April 10, 2012 --- Due: Thursday, May 3, 2012
DUE DATE FOR REPORT AND PRESENTATION: Wednesday December 4, 2019 \\
Last compiled: \today
\end{center}

%------------------------

\subsection*{Instructions}

For your final project, you will explore a particular inverse problem or a technical aspect of inverse theory.
%
\begin{itemize}
\item The final project is worth 15\% of your final grade.
\item A proposed title and draft abstract are due (by email) on Wednesday November 6.
\item The final project is due in the form of an in-class presentation on and also a report---both on Wednesday December 4.

\item Final project guidelines.
%
\begin{enumerate}
\item The project should contain:
%
\begin{enumerate}
\item a review of essential literature on the topic
\item a moderate level of applied (computational) analysis, either through modeling of real (or synthetic) data, or by demonstrating some of the key relationships in your inverse problem or topic
\end{enumerate}

\item The project should address the following questions:
%
\begin{enumerate}
\item What are the unknown model parameters?
\item What are the observations?
\item What is the forward model? What is the underlying physics?
\item What approximations are used in the forward model?
\item What measurement is being made?
\item What is the misfit function? (What norm is used? What weighting factors are used?)
\item What is the prior information (or regularization)?
\item What is the inverse method? (Is the gradient or Hessian used?)
\item What is $\covd$?
\item What are the uncertainties (and covariances) of the posterior model parameters?
\item What step of your inverse problem accounts for the most computational cost?
\item What is the practical significance of your topic? What problems or scientific questions is your problem related to?
\end{enumerate}
%
Your particular focus might be on just one of these questions, for example.

\item The products of the project are:
%
\begin{enumerate}
%\item a 15-minute presentation (with visuals) summarizing your project
\item a 5--10-minute presentation with no more than 4 slides
\item a written report with no more than 6 pages of single-spaced text (not including references) and 4 pages of figures \\
{\bf Turn in the report as a hard copy and also as a pdf via email.}
\end{enumerate}

\item The project will be evaluated with $90\%$ on the report and $10\%$ on the presentation.

\end{enumerate}

\item You are encouraged to choose a project directly related to your field of research. However, {\bf be careful to choose a reasonable scope}, keeping in mind you must address all the points above within $\sim$3 weeks of outside-of-class time.

\end{itemize}

%------------------------

\pagebreak
\subsection*{Example project topics}

You are welcome to propose a topic to me. Here are some possibilities:
%
\begin{itemize}
\item Characterization of fault slip models from coseismic InSAR data.

% Synthetic Interferogram Calculator
% http://sioviz.ucsd.edu/~fialko/software.html

\item Inferring river deposition (current or ancient) from zircon age spectra of source rocks and sediment samples.

\item Inferring basal flow of glaciers from surface velocity fields.

%\item Image processing and analysis (deblurring, denoising, etc).

\item Inferring a source function based on measurements of dispersed contaminants (\eg 2010 BP oil spill in Gulf of Mexico).

\item Inferring earthquake source properties (source-time function, hypocenter, radiation pattern) from seismic waveforms.

\item Seismic tomography, medical imaging, etc.

\item Singular value decomposition (or principal component analysis) of a large data set

\item Exploration of the variability of finite slip models for large earthquakes.

\verb+http://www.seismo.ethz.ch/static/srcmod/+

\item Implementation and exploration of 3D Gaussian random fields in Matlab.

%\item Chena Ridge blast
%\item covariance of topography

%-------------------------

\item Take an inverse problem that is published in the literature and thoroughly explore all facets of the problem. Compare at least three different published approaches to the problem.

\end{itemize}
%
In thinking about a topic, consider the three critical components of an inverse problem: (1)~observations, (2)~unknown model parameters, and (3)~a ``forward model'' that relates model parameters to predictions (or observations).

Browse some articles in the journal {\em Inverse Problems} to get some ideas.

%-------------------------------------------------------------
\bibliographystyle{agu08}
\bibliography{preamble,refs_carl,REFERENCES}
%-------------------------------------------------------------


%-------------------------------------------------------------
\end{document}
%-------------------------------------------------------------
