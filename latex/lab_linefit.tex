% dvips -t letter lab_linefit.dvi -o lab_linefit.ps ; ps2pdf lab_linefit.ps
\documentclass[11pt,titlepage,fleqn]{article}

\usepackage{amsmath}
\usepackage{amssymb}
\usepackage{latexsym}
\usepackage[round]{natbib}
%\usepackage{epsfig}
\usepackage{graphicx}
\usepackage{bm}

\usepackage{url}
\usepackage{color}

%--------------------------------------------------------------
%       SPACING COMMANDS (Latex Companion, p. 52)
%--------------------------------------------------------------

\usepackage{setspace}    % double-space or single-space
\usepackage{xspace}

\renewcommand{\baselinestretch}{1.2}

\textwidth 460pt
\textheight 690pt
\oddsidemargin 0pt
\evensidemargin 0pt

% see Latex Companion, p. 85
\voffset     -50pt
\topmargin     0pt
\headsep      20pt
\headheight   15pt
\headheight    0pt
\footskip     30pt
\hoffset       0pt

\include{carlcommands}

\graphicspath{
  {./figures/}
}

\newcommand{\repodir}{{\tt inverse}}

\newcommand{\howmuchtime}{Approximately how much time {\em outside of class and lab time} did you spend on this problem set? Feel free to suggest improvements here.}

% provide space for students to write their solutions
\newcommand{\vertgap}{\vspace{1cm}}

\newcommand{\Ucolor}{\textcolor{red}{\bU}}
\newcommand{\Vcolor}{\textcolor{blue}{\bV}}

\newcommand{\Gcolor}{\textcolor{red}{\bU}\bS\textcolor{blue}{\bV^T}}
\newcommand{\Gpcolor}{\textcolor{red}{\bU_p}\,\bS_p\textcolor{blue}{\bV_p^T}}
\newcommand{\Gdcolor}{\textcolor{blue}{\bV_p}\,\bS_p^{-1}\textcolor{red}{\bU_p^T}}
\newcommand{\GcolorT}{\textcolor{blue}{\bV}\bS^T\textcolor{red}{\bU^T}}
\newcommand{\GpcolorT}{\textcolor{blue}{\bV_p}\,\bS_p\textcolor{red}{\bU_p^T}}
\newcommand{\GdcolorT}{\textcolor{red}{\bU_p}\,\bS_p^{-1}\textcolor{blue}{\bV_p^T}}


\renewcommand{\baselinestretch}{1.0}

% change the figures to ``Figure L3'', etc
\renewcommand{\thefigure}{L\arabic{figure}}
\renewcommand{\thetable}{L\arabic{table}}
\renewcommand{\theequation}{L\arabic{equation}}
\renewcommand{\thesection}{L\arabic{section}}

%--------------------------------------------------------------
\begin{document}
%-------------------------------------------------------------

\begin{spacing}{1.2}
\centering
{\large \bf Lab Exercise: Fitting a line to scattered data [linefit]} \\
GEOS 626: Applied Seismology, Carl Tape \\
GEOS 627: Inverse Problems and Parameter Estimation, Carl Tape \\
Last compiled: \today
\end{spacing}

%------------------------

\vspace{-0.5cm}
\subsection*{Overview}

The objective of this lab is to introduce the concepts of forward and inverse problems by exploring the example of fitting a line to scattered data. We also want to develop some basic capabilities in Matlab, such as vector-matrix operations and plotting.

\begin{enumerate}
\item Write down the forward model $d_i = m_1 + m_2 x_i$ in schematic matrix form $\bd = \bG\bem$.

\vertgap
\vertgap

\item Show that minimizing the least squares misfit function (\ie sum of squares of residuals) leads to the solution $\bem = (\bG^T\bG)^{-1}\bG^T\dvec$ (see \verb+notes_taylor.pdf+).

\vertgap
\vertgap
\vertgap

\item Run the script \verb+lab_linefit.m+ and identify the key parts of the code associated with the forward problem and the inverse problem. What is the Matlab syntax for $\bem^T$, $\bG\bem$, and $\bG^T\bd$?

\vertgap

\item How are the ``target'' (or ``true'') data generated?

\vertgap

\item What are the different ways in Matlab to compute the solution vector $\bem$? \\
(Type \verb+help \+ or \verb+help inv+ for details.)

\vertgap

\item How does the estimated model $\bem_{\rm est}$ compare with the target model $\bem_{\rm tar}$?

\vertgap

How does the standard deviation of the residuals compare with the assumed value of $\sigma$?

\vertgap

\item What does the histogram show? \\
How does it change if you increase the number of points to $10^5$? \\
How does it change if you set $\sigma = 0$?

\vertgap

\item Re-set \verb+ndata=50+ and $\sigma = 0.3$ and add one outlier to the simulated observations. This can be done by replacing the first observation point with a value that is $1000\sigma$ larger. What happens (and why)?

\vertgap

%\pagebreak
\item Now comment out the outlier. Comment out the \verb+break+ statement, then run the last block of code. Make sure you understand what is being plotted and how it is calculated.
%
\begin{itemize}
\item What is the dimension of model space?
\item Fill out the table below.
\item In the code, why is the operation \verb+res.*res+ used instead of \verb+res*res+, \verb+res'*res+, or \verb+res*res'+?
\end{itemize}

\vertgap
\begin{tabular}{c|c|c|c}
\hline
variable & matlab variable  & dimension      & description \\ \hline\hline
         & \verb+dobs+     &                &              \\ \hline
         & \verb+G+        &                &              \\ \hline
$\misfitvar(\bem)$  & --              &                &              \\ \hline
---      & \verb+RSSm+     &                &              \\ \hline
\bem     & \verb+mtry+     & $2 \times 1$   & \hspace{3cm} \\ \hline
         & \verb+dtry+     &                &              \\ \hline
         & \verb+res+      &                &              \\ \hline
\bgamma(\bem)  & --        & $2 \times 1$   & gradient of misfit function \\ \hline
--       & \verb+gammam+   & $2 \times n_g$ &  \\ \hline
\end{tabular}

\item Referring to the class notes ``Taylor series and least squares'' (\verb+notes_taylor.pdf+) compute the gradient $\bgamma(\bem)$ for the same grid of $\bem$ that was used to plot the misfit function.

Hint: one line of code (plus one line to initialize $\bgamma(\bem)$)

Matlab's \verb+gradient+ function will not help you here, since you need an exact evaluation of the gradient, whereas Matlab will provide a numerical gradient for a grid of pre-computed values.

\vertgap

\item Using the \verb+quiver+ command, superimpose the vector field $-\gamma(\bem)$ on the contour plot of the misfit function.

Set \verb+nx = 10+ so that the vector field is more visible. \\
Hint: This is one line of code. \\
What is the relationship between the contours of $F(\bem)$ and $\bgamma(\bem)$?

\vertgap

How does $\|\bgamma(\bem)\|$ vary with respect to $\bem_{\rm est}$? \\
% it gets larger 
Bonus: Use \verb+surf+ to plot $\|\bgamma(\bem)\|$. What shape is it?
% a cone

\vertgap

\item What is the dependence of the Hessian on the model, $\bH(\bem)$? \\
% Hessian does not depend on m
What does that imply about $F(\bem)$?
% it is strictly quadratic

%What does that imply about the forward model $\bg(\bem)$?

\end{enumerate}

%-------------------------------------------------------------
\end{document}
%-------------------------------------------------------------
