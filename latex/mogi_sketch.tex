% ORIGINAL: caption next to figure
%\begin{figure}
%\begin{tabular}{ccc}
%\parbox{6cm}{
%\caption[]
%{{
%Sketch showing the relationship between the surface displacement vector $\bu$ and the line-of-sight vector, $\bLh$. The satellite flies in the track direction at angle $t$ with respect to north; in the sketch, this is the in-to-page direction. For the outward deformation signal depicted here, the projection vector, $(\bu^T\bLh)\bLh$, points to the satellite with a {\em negative} line-of-sight distance, $\bu^T\bLh$.
%\label{proj}
%}}
%}
%&
%\parbox{9cm}{\includegraphics[width=8cm]{proj_sketch_LOS.eps}}
%\end{tabular}
%\end{figure}
 
\begin{figure}
\centering
\includegraphics[width=16cm]{proj_sketch_LOS.eps}
\caption[]
{{
Sketch showing the conventions for the angles and vectors.
(left) Map view showing satellite flight path. The satellite flies in the track direction at angle $t$ with respect to north.
% NOTE THAT L-HAT SHOULD REALLY BE L-HAT sin(l), so that when l=90 this gives L-HAT and when l=0 this gives ZERO.
(right) Relationship between the surface displacement vector $\bu$ and the look vector, $\bLh$.
For the outward deformation signal depicted here, the projection vector $(\bu^T\bLh)\bLh$ points to the satellite with a {\em negative} look displacement $\bu^T\bLh$.
The look vector is also known as the line-of-sight vector.
\label{proj}
}}
\end{figure}
