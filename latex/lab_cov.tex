% dvips -t letter lab_cov.dvi -o lab_cov.ps ; ps2pdf lab_cov.ps
\documentclass[11pt,titlepage,fleqn]{article}

\usepackage{amsmath}
\usepackage{amssymb}
\usepackage{latexsym}
\usepackage[round]{natbib}
%\usepackage{epsfig}
\usepackage{graphicx}
\usepackage{bm}

\usepackage{url}
\usepackage{color}

%--------------------------------------------------------------
%       SPACING COMMANDS (Latex Companion, p. 52)
%--------------------------------------------------------------

\usepackage{setspace}    % double-space or single-space
\usepackage{xspace}

\renewcommand{\baselinestretch}{1.2}

\textwidth 460pt
\textheight 690pt
\oddsidemargin 0pt
\evensidemargin 0pt

% see Latex Companion, p. 85
\voffset     -50pt
\topmargin     0pt
\headsep      20pt
\headheight   15pt
\headheight    0pt
\footskip     30pt
\hoffset       0pt

\include{carlcommands}

\graphicspath{
  {./figures/}
}

\newcommand{\repodir}{{\tt inverse}}

\newcommand{\howmuchtime}{Approximately how much time {\em outside of class and lab time} did you spend on this problem set? Feel free to suggest improvements here.}

% provide space for students to write their solutions
\newcommand{\vertgap}{\vspace{1cm}}

\newcommand{\Ucolor}{\textcolor{red}{\bU}}
\newcommand{\Vcolor}{\textcolor{blue}{\bV}}

\newcommand{\Gcolor}{\textcolor{red}{\bU}\bS\textcolor{blue}{\bV^T}}
\newcommand{\Gpcolor}{\textcolor{red}{\bU_p}\,\bS_p\textcolor{blue}{\bV_p^T}}
\newcommand{\Gdcolor}{\textcolor{blue}{\bV_p}\,\bS_p^{-1}\textcolor{red}{\bU_p^T}}
\newcommand{\GcolorT}{\textcolor{blue}{\bV}\bS^T\textcolor{red}{\bU^T}}
\newcommand{\GpcolorT}{\textcolor{blue}{\bV_p}\,\bS_p\textcolor{red}{\bU_p^T}}
\newcommand{\GdcolorT}{\textcolor{red}{\bU_p}\,\bS_p^{-1}\textcolor{blue}{\bV_p^T}}


%\pagestyle{empty}

% change the figures to ``Figure L3'', etc
\renewcommand{\thefigure}{L\arabic{figure}}
\renewcommand{\thetable}{L\arabic{table}}
\renewcommand{\theequation}{L\arabic{equation}}
\renewcommand{\thesection}{L\arabic{section}}

%--------------------------------------------------------------
\begin{document}
%-------------------------------------------------------------

\begin{spacing}{1.2}
\begin{center}
{\large \bf Lab Exercise: Samples of a covariance matrix} \\
%GEOS 626: Applied Seismology, Carl Tape \\
GEOS 627: Inverse Problems and Parameter Estimation, Carl Tape \\
Last compiled: \today
\end{center}
\end{spacing}

%------------------------

\subsection*{Problem}

See class notes \verb+notes_tarantola.pdf+ for background.

\begin{enumerate}
\item Consider the two vectors
%
\begin{equation*}
\ba =  \left[ \begin{array}{r} 1 \\ 2 \\ 3 \end{array} \right],
\quad\quad
\bb =  \left[ \begin{array}{r} 1 \\ 2 \\ 3 \\ 4 \\ 5 \end{array} \right]
\end{equation*}
%
Calculate the following by hand:
%
\begin{itemize}
\item $\ba^T \ba$
\item $\ba \ba^T$
\item $\bb^T \bb$
\item $\bb \bb^T$
\item $\ba^T \bb$
\item $\ba \bb^T$
\item $\bb^T \ba$
\item $\bb \ba^T$
\end{itemize}

\item Use \verb+randn+ to generate $P$ samples of $M \times 1$ vectors $\bw$ that have unit covariance, $\bC = \bI$.
This means that the entries of $\bw$ will have mean $\mu \approx 0$ and standard deviation $\sigma \approx 1$. \\
Use $P = 500$ and $M = 100$.

It may help to think of sample $\bw_p$ as a profile of a topographic transect or as a seismogram; in any case, there are $M$ discretization points per sample. You have $P$ samples.

\item Using \verb+plot_histo.m+, plot histograms of 12 or so $\bw$ to convince yourself that each $\bw$ appears to be a Gaussian distribution with $\mu \approx 0$ and $\sigma \approx 1$. (You can increase $M$ to make this more clear.)

\item Calculate the mean sample $\bar{\bw}$ for the set of $P$ samples. Make sure it is $M \times 1$.

\item Using a for loop, calculate the estimated covariance matrix \verb+Csamp+ for the set of $P$ samples.

\item Check that your \verb+Csamp+ matches what you get when you use Matlab's \verb+cov+ function.

\item Plot your estimated covariance matrix using \verb+imagesc(Csamp)+. \\
To make the comparison, use these commands for each plot: \\
\verb+axis equal, axis tight, caxis([0 1]), colorbar+

\item What is the theoretical covariance matrix? Plot it using the same settings.

\item Plot the residual between the two matrices. Use \verb+caxis(0.1*[-1 1])+ \\
Is the residual matrix symmetric? Why or why not?
% R = C - I, R^T = (C - I)^T = C^T - I^T = C^T - I
% C = C^T by applying the transpose to the expression in notes_tarantola.pdf

\item Explain why the matrix-square-root method would give the same kinds of samples as the method above. In other words, given an $M \times M$ covariance matrix $\bC$, how can you generate an $M \times M$ representative sample $\bx$? (And what happens if $\bC = \bI$?)
% notes_tarantola.pdf: x = Lw, so if C = LL^T and C=I, then L=I and x = w

\end{enumerate}

%-------------------------------------------------------------
\end{document}
%-------------------------------------------------------------
